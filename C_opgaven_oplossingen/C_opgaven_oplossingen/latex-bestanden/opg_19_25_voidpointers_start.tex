\beginoef
Gegeven onderstaand hoofdprogramma  (haal eerst het latex-bestand af van Minerva, zodat je geen code hoeft over te tikken) , en de functies \verb}som}, \verb}product} en \verb}verschil}.
Je ziet dat de derde tabel wordt ingevuld aan de hand van de twee eerste tabellen. Afhankelijk van de 
laatste parameter van de procedure \verb}vul_tabel}, bevat de derde tabel de som, respectievelijk product of verschil van de overeenkomstige elementen
uit de eerste twee tabellen.
\begin{footnotesize}
\begin{minipage}[t]{9cm}
\begin{verbatim}
#include <stdio.h>
#define AANTAL 5
int som(int a, int b){
    return a+b;
}
int product(int a, int b){
    return a*b;
}
int verschil(int a, int b){
    return a-b;
}
void schrijf(const int * t, int aantal){	
    int i;
    for(i=0; i<aantal; i++){
        printf("%i ",t[i]);
    }
    printf("\n");
}
\end{verbatim}
\end{minipage}
\begin{minipage}[t]{8cm}
\begin{verbatim}
int main(){
    int a[AANTAL];
    int b[AANTAL];
    int c[AANTAL];
    int i;
    for(i=0; i<AANTAL; i++){
        a[i] = 10*i;
        b[i] = i;
    }
	
    vul_tabel(a,b,c,AANTAL,som);
    schrijf(c,AANTAL);
	
    vul_tabel(a,b,c,AANTAL,product);
    schrijf(c,AANTAL);
	
    vul_tabel(a,b,c,AANTAL,verschil);
    schrijf(c,AANTAL);
    return 0;
}
\end{verbatim}
\end{minipage}
\end{footnotesize} Als output zal er dus verschijnen:
\begin{verbatim}
0 11 22 33 44
0 10 40 90 160 
0 9 18 27 36 
\end{verbatim}
Schrijf de procedure \verb}vul_tabel(...)}. De laatste parameter is een functie. 
Definieer de procedure \verb}vul_tabel} n\'a je \verb}main}-functie. Dan moet je \verb}vul_tabel}
vooraf declareren. Laat in die parameterlijst de benamingen van de parameters zelf weg, schrijf enkel de types neer.

%Merk op: in deze oefening hebben we nog geen void-pointers nodig. De reden? Elke functie (\verb}som}, \verb}product}, \verb}verschil}) heeft hetzelfde
%returntype \'en neemt ook telkens twee parameters van hetzelfde type. Dus is er geen behoefte om deze types hier te vervangen door een overkoepelend type \verb}void*}.
\endoef

