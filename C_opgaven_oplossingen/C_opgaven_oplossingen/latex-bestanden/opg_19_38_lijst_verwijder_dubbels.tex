\beginoef
\begin{footnotesize}
\begin{verbatim}
int main(){
    srand(time(NULL));
    knoop * l = maak_gesorteerde_lijst_automatisch(10,100);
    print_lijst(l);
    printf("\nnu worden dubbels verwijderd: \n");
    verwijder_dubbels(...); /* aan te vullen */
    printf("\nna verwijderen van dubbels: \n\n");
    print_lijst(l);	
    ... /* aan te vullen */
    return 0;
}
\end{verbatim}
\end{footnotesize}
In bovenstaand hoofdprogramma wordt er automatisch een stijgend gesorteerde lijst aangemaakt, waar dubbele elementen in kunnen zitten. Implementeer de functie
\\\verb}maak_gesorteerde_lijst_automatisch(aantal,bovengrens)}. De eerste parameter geeft aan hoe\-veel elementen de lijst moet bevatten. De tweede parameter geeft
aan wat het grootste getal zal zijn. Een tip: bouw de lijst op door vooraan telkens een iets kleiner getal dan het vorige toe te voegen. Daarvoor genereer je
een getal uit de verzameling $\{0,1,2\}$ met de functie \verb}rand()} uit \verb}stdlib.h} en trek je dit getal af van het vorige toegevoegde getal (of van het getal \verb{bovengrens{ bij het toevoegen van de eerste knoop).

Schrijf de procedure \verb}verwijder_dubbels(...)} die alle dubbels uit de gelinkte lijst verwijdert. Als enige parameter
geef je de gelinkte lijst mee. Vul het hoofdprogramma verder aan zoals het hoort.
\endoef
