\beginoef
Wat wordt er uitgeschreven? Uiteraard beantwoord je deze vraag zonder computer. 
\\Bijvraag: hoe zou je de hardgecodeerde bovengrenzen van de drie for-lussen kunnen 
vervangen door zelf de grootte van de array's te bepalen? 
%\newpage

\begin{footnotesize}
\begin{minipage}{8cm}
\begin{verbatim}

#include <stdio.h>
int main(){
    int t[6] = {0,10,20,30,40,50};
    int* pt[3];
	
    int i;
    for(i=0; i<3; i++){
        pt[i] = &t[2*i];	
    }
	
    pt[1]++;
    pt[2] = pt[1];
    *pt[1] += 1;
    *pt[2] *= 2;
\end{verbatim}
\end{minipage}
\begin{minipage}{8cm}
\begin{verbatim}
	
    int ** ppt = &pt[0];
    (*ppt)++;          
    **ppt += 1;
		
    for(i=0; i<6; i++){
        printf("%d ",t[i]);
    }
    printf("\n");
    for(i=0; i<3; i++){
        printf("%d ",*pt[i]);	
    }    
    printf("\n");
    return 0;
}
\end{verbatim}
\end{minipage}
\end{footnotesize}
\endoef
