\beginoef
\begin{enumerate}
\item 
Schrijf een functie \verb}lees()} die een lijn (bestaande uit 1 of meerdere woorden) inleest vanop het klavier, en een nieuwe C-string teruggeeft. 
Je weet niet hoe lang de tekst is, maar na oproep van de functie neemt de C-string niet meer geheugenplaats in dan strikt noodzakelijk. 
Indien de tekst langer is dan 1000 karaktertekens, breek je het af na het 1000$^e$ teken en kuis je de resterende tekst op. Test dit uit door de constante 1000 aan te passen - uiteraard.
\\Zorg er ook voor dat op het einde van de resulterende C-string geen newline-karakter staat.
\item
Test je functie uit in het onderstaande (half-afgewerkte) hoofdprogramma (haal eerst het latex-bestand af van Minerva, zodat je geen code hoeft over te tikken). 
\\Vervolledig het hoofdprogramma met 1 regel code.
\end{enumerate}
\begin{verbatim}
int main(){
    int i;
    for(i=0; i<5; i++){    
        char * tekst = lees();
        printf("Ik las in %s.\n",tekst);
    }	
    return 0;
}
\end{verbatim}

\endoef

