\beginoef
%
Gegeven onderstaand hoofdprogramma  (haal eerst het latex-bestand af van Minerva, zodat je geen code hoeft over te tikken). Schrijf de nodige code om dit te laten werken; gebruik overal {\bf verplicht schuivende pointers}. De procedure \verb}schrijf(begin,eind)} werd al gevraagd in oefening 21.


\begin{enumerate}
%Marleen : volgorde omgewisseld - laatste zin aangepast.
\item de functie \verb}pointerNaarEersteKleineLetter(p)} geeft een pointer terug naar de eerste kleine letter die te vinden is vanaf de huidige positie van de pointer \verb}p}.
Indien er geen kleine letters meer volgen, staat \verb}p} op het einde van de c-string (=voorbij de laatste letter).

\item de procedure \verb}verzetNaarEersteHoofdletter(p)} verzet de gegeven pointer \verb}p} zodat hij wijst naar de eerste hoofdletter die vanaf zijn huidige positie te vinden is.
(Ook hier: voorzie de situatie waarbij er geen hoofdletters meer volgen.)

\end{enumerate}
\begin{footnotesize}
\begin{verbatim}
int main(){	
    const char zus1[50] = "sneeuwWITje";
    const char zus2[50] = "rozeROOD";                                        
    const char* begin;
    const char* eind;	   
    begin = zus1;
    verzetNaarEersteHoofdletter(&begin);	
    eind = pointerNaarEersteKleineLetter(begin);		
    schrijf(begin,eind);   /* schrijft 'WIT' uit */
    printf("\n");	    
    begin = zus2;
    verzetNaarEersteHoofdletter(&begin);
    eind = pointerNaarEersteKleineLetter(begin);	
    schrijf(begin,eind);   /* schrijft 'ROOD' uit */	
    return 0;
}
\end{verbatim}
\end{footnotesize}
\endoef

