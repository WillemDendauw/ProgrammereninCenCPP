        
\beginoef
Dit is een vervolg op oefening 13. Gegeven onderstaand  hoofdprogramma (haal eerst het latex-bestand af van Minerva, zodat je geen code hoeft over te tikken). 
\begin{verbatim}
#include <stdio.h>
int main(){
	
    char letters [] = {'p','o','r','g','o','e','d','o','i','e','o','k',
                       'i',':','a','-','t','('};
    const char *p = letters; 

    schrijf_aantal(letters+3,4);
    printf("\n");
    schrijf(p+10,p+12);
    
} 
\end{verbatim} 

\begin{enumerate}
\item Kan je de lengte van de array opvragen via de pointer \verb{p{?
\\Ga na dat de geheugenruimte, ingenomen door een pointer altijd gelijk is (onafhankelijk van het type waarnaar de pointer verwijst).

\item Schrijf de procedure \verb}schrijf_aantal} die het aantal gevraagde letters uitschrijft vanaf de pointer opgegeven in de eerste parameter.

\item Schrijf de procedure \verb}schrijf(begin,eind)} die de letters uitschrijft die te vinden zijn tussen de plaats waar de pointers \verb}begin} en \verb}eind} naar wijzen. (Laatste grens niet inbegrepen; je mag er vanuit gaan dat beide pointers wijzen naar elementen in dezelfde array van karakters, en dat \verb}begin < eind} .)

\end{enumerate}
\endoef



	
