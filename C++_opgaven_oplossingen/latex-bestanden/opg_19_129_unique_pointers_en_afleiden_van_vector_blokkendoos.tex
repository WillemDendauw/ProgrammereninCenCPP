\beginoef
%\begin{figure}[H]
%\centering
%\includegraphics[width=0.2\linewidth]{blokkendoos.jpg}
%\end{figure}

Gegeven een blokkendoos waarin je figuren kan steken zoals in een vector, met \'e\'en verschil: de grootste van deze figuren (wat oppervlakte betreft) 
zal altijd apart bewaard worden. Vraag je de grootste figuur van de blokkendoos, dan verdwijnt die ook effectief uit de doos.

Om zeker te zijn elke figuur maar 1 keer worden toegekend gebruik je \verb{unique pointer{.

Download de bestanden \verb{figuren.h{ en \verb{figuren.txt{ van Minerva.

Bekijk grondig de interface van de klasse \verb{Figuur{ en de inhoud van het bestand \verb{figuren.txt{.
Bekijk daar het gegeven hoofdprogramma. Probeer eerst zonder de tips te voorspellen wat je
moet implementeren. Pas daarna lees je verder wat onder de code verklapt wordt.

\begin{footnotesize}
\begin{verbatim}
#include "figuren.h"
#include <memory>

class Blokkendoos : vector<unique_ptr<Figuur>>{
  private:
    unique_ptr<Figuur> max_opp;
    void schrijf(ostream&)const;
  public:
    Blokkendoos();
    Blokkendoos(const string & bestandsnaam);
    unique_ptr<Figuur> geef_figuur_met_grootste_oppervlakte();	
    void push_back(unique_ptr<Figuur>& figuur);
	
    friend ostream& operator<<(ostream& out, const Blokkendoos& l){
        l.schrijf(out);
        return out;	
    }	
};

int main(){
    Blokkendoos blokkendoos("figuren.txt"); 
    cout<<endl<<"ALLE FIGUREN: ";
    cout<<blokkendoos<<endl;
	
    cout<<endl<<"DE 3 GROOTSTE, van groot naar klein: "<<endl;
    for(int i=0; i<3; i++){
        cout<<"figuur met grootste opp:    "<<*blokkendoos.geef_figuur_met_grootste_oppervlakte()<<endl;
    }

    cout<<endl<<"DE NIEUWE BLOKKENDOOS BEVAT ALLEEN NOG DE KLEINERE FIGUREN: ";
    cout<<blokkendoos<<endl;
	
	
    return 0;	
}

\end{verbatim}
\end{footnotesize}
\newpage
{\bf Tips}

\begin{enumerate}
\item Bij het toevoegen van een figuur \verb}fig} aan de blokkendoos (\verb}push_back}) maak je onderscheid tussen twee gevallen. Indien de grootste figuur nog
niet gekend is (dan is de vector dus zeker nog leeg), initialiseer je deze grootste figuur met de figuur \verb}fig}. In het andere geval
voeg je de nieuwe figuur achteraan de vector toe. Indien die echter (qua oppervlakte) groter blijkt te zijn dan \verb}max_opp}, dan wissel je beide.
\item De methode die de grootste figuur teruggeeft, moet er ook voor zorgen dat de blokkendoos een nieuwe grootste figuur aanduidt uit de voorraad die nog in de vector zit.
Pas de grootte van de vector ook aan!
\end{enumerate}
\endoef

