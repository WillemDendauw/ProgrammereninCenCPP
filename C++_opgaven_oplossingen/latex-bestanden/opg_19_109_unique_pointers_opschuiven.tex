\beginoef

Gegeven onderstaande code.
\begin{footnotesize}
\begin{verbatim}
#include <memory>
#include <iostream>
using namespace std;

void schrijf(const string * s, int aantal){
    cout<<endl;
    for(int i=0; i<aantal-1; i++){
        cout<<s[i]<<" - ";
    }
    cout<<s[aantal-1];
}

void verwijder(string * s, int aantal, int volgnr){
   for(int i = volgnr; i < aantal-1; i++){
       s[i] = s[i+1];	
   }  
   s[aantal-1] = "";  //laatste element leeg maken
}

int main(){
	
    string namen[]={"Rein","Ada","Eppo" , "Pascal" , "Ilse"};
	
    schrijf(namen,5);
    verwijder(namen,5,1);
    schrijf(namen,5); //alle namen tonen
	
    return 0;
}
\end{verbatim}
\end{footnotesize}
Test uit wat het programma doet. Het laatste element is leeg, maar wordt nog steeds getoond (dat is de bedoeling!).

{\bf Probleem:} De code \verb}s[i] = s[i+1];} zal de string {\bf kopi\"eren}.
Dat moeten we vermijden, want een string kan in principe heel groot zijn. 

{\bf Opdracht:} Schrijf twee nieuwe procedures, die bij het onderstaande hoofdprogramma horen.
We bewaren nu (unique) pointers in de array, zodat we bij het opschuiven van de elementen in de array
enkel pointers moeten verleggen, en geen kopie\"en maken.
\begin{footnotesize}
\begin{verbatim}
int main(){
    unique_ptr<string> pnamen[]={...}; //vul zelf deze array aan met 5 unieke pointers
    schrijf(pnamen,5);
    verwijder(pnamen,5,1);
    schrijf(pnamen,5);
	
    return 0;
}
\end{verbatim}
\end{footnotesize}
Probeer ook eens het laatste element uit de array te ``verwijderen''!

\endoef
