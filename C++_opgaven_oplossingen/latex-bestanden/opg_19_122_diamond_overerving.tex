\beginoef
Vul de klassen \verb}Rechthoek}, \verb}GekleurdeRechthoek}, \verb}Vierkant} en \verb}GekleurdVierkant} aan.
Gebruik overerving waar het kan: een \verb}GekleurdVierkant} erft zowel van \verb}GekleurdeRechthoek} als van \verb}Vierkant}.
Zorg dat het hoofdprogramma werkt en EXACT de gegeven output oplevert: kijk goed na, het venijn zit 'm in de details!
\begin{footnotesize}
\begin{verbatim}
#include <iostream>
using namespace std;

class Rechthoek {
 public:
   Rechthoek();
   Rechthoek(int, int);
   ...
   // attributen voorzien voor hoogte en breedte
   // (type int)
};

// afgeleid van Rechthoek; pas aan in hoofding
class GekleurdeRechthoek {
 public:
   ...
   // enkel extra attribuut voor kleur
};

// afgeleid van Rechthoek; pas aan in hoofding
class Vierkant {
    ...
    // geen extra attributen voorzien!
};

// afgeleid van GekleurdRechthoek en Vierkant; pas aan in hoofding
class GekleurdVierkant {
    ...
    // geen extra attributen voorzien!
};

int main () {
   Rechthoek r1;
   r1.print(cout);
   cout << "  oppervlakte: " << r1.oppervlakte() << endl
        << "  omtrek: " << r1.omtrek() << endl;
        
   Rechthoek r2(4,6);
   cout << r2;
   cout << "  oppervlakte: " << r2.oppervlakte() << endl
        << "  omtrek: " << r2.omtrek() << endl;
        
   GekleurdeRechthoek gr1;
   gr1.print(cout);
   cout << "  oppervlakte: " << gr1.oppervlakte() << endl
        << "  omtrek: " << gr1.omtrek() << endl;
        
   GekleurdeRechthoek gr2(5,7);
   cout << gr2;
   cout << "  oppervlakte: " << gr2.oppervlakte() << endl
        << "  omtrek: " << gr2.omtrek() << endl;
        
   GekleurdeRechthoek gr3(6,9,"rood");
   gr3.print(cout);
   cout << "  oppervlakte: " << gr3.oppervlakte() << endl
        << "  omtrek: " << gr3.omtrek() << endl; 
		
   Vierkant v1;
   cout << v1;
   cout << "  oppervlakte: " << v1.oppervlakte() << endl
        << "  omtrek: " << v1.omtrek() << endl;
        
   Vierkant v2(10);
   v2.print(cout);
   cout << "  oppervlakte: " << v2.oppervlakte() << endl
        << "  omtrek: " << v2.omtrek() << endl; 
		
   GekleurdVierkant gv1;
   cout << gv1;   
   cout << "  oppervlakte: " << ...
        << "  omtrek: " << ...
        
   GekleurdVierkant gv2(12);
   gv2.print(cout);
   cout << "  oppervlakte: " << ...
        << "  omtrek: " << ...
        
   GekleurdVierkant gv3(15,"geel");
   cout << gv3;
   cout << "  oppervlakte: " << ...
        << "  omtrek: " << ... 
   return 0;
}
\end{verbatim}
\end{footnotesize}
De gevraagde output is:
\begin{footnotesize}
\begin{verbatim}
Rechthoek: 1 op 1
  oppervlakte: 1
  omtrek: 4
Rechthoek: 6 op 4
  oppervlakte: 24
  omtrek: 20
Rechthoek: 1 op 1
  kleur: onbekend
  oppervlakte: 1
  omtrek: 4
Rechthoek: 7 op 5
  kleur: onbekend
  oppervlakte: 35
  omtrek: 24
Rechthoek: 9 op 6
  kleur: rood
  oppervlakte: 54
  omtrek: 30
Vierkant: zijde 1
  oppervlakte: 1
  omtrek: 4
Vierkant: zijde 10
  oppervlakte: 100
  omtrek: 40
Vierkant: zijde 1
  kleur: onbekend
  oppervlakte: 1
  omtrek: 4
Vierkant: zijde 12
  kleur: onbekend
  oppervlakte: 144
  omtrek: 48
Vierkant: zijde 15
  kleur: geel
  oppervlakte: 225
  omtrek: 60
\end{verbatim}
\end{footnotesize}
Merk op: de implementatie van de klassen moet zo beknopt mogelijk zijn,
dus maak optimaal gebruik van overerving en overschrijf enkel de methodes die echt nodig zijn.
\endoef

