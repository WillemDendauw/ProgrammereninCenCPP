
\beginoef
In oefening 25 werd een functie als parameter doorgegeven voor een andere functie of procedure. In C (en C++) moet je die functie eerst expliciet een naam geven en implementeren.

In C++11 kan het in \'e\'en moeite: met $\lambda$-functies maak je de functie 'on-the-fly' aan.

Pas onderstaand hoofdprogramma aan: 
\begin{itemize}
\item schrijf de procedure \verb}schrijf}.
\item schrijf de procedure \verb}vul_array}. .
\\Let op, het type van de vierde parameter is nu geen functiepointer, want je werkt in C++11/14 in plaats van C!
\item vervolledig de aanroep van de methode \verb}vul_array}: vervang \verb}......} door een  $\lambda$-functie. 

\end{itemize}
\begin{footnotesize}
\begin{verbatim}
int main(){
     const int GROOTTE = 10;
     int a[] = {0,1,2,3,4,5,6,7,8,9};
     int b[] = {0,10,20,30,40,50,60,70,80,90};
     int c[GROOTTE];
          
     vul_array(a,b,c,GROOTTE,........);
     schrijf("SOM:      ",c,GROOTTE);

     vul_array(a,b,c,GROOTTE,........);
     schrijf("PRODUCT:  ",c,GROOTTE);

     vul_array(a,b,c,GROOTTE,........);
     schrijf("VERSCHIL: ",c,GROOTTE);
     
     return 0;
}
\end{verbatim}
\end{footnotesize}
\endoef
