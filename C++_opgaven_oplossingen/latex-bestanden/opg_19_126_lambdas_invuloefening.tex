\beginoef
Vul het  onderstaande hoofdprogramma aan met lambdafuncties en implementeer de lidfunctie \verb}geef_extremum} van de klasse \verb}Groep}.

\begin{footnotesize}
\begin{verbatim}
#include <iostream>
#include <vector>
#include <string>
using namespace std;

class Persoon{
    public:
       string voornaam;
       string naam;
       int leeftijd;
       Persoon(const string & v, const string & n, int l):voornaam(v),naam(n),leeftijd(l){}
};	

ostream& operator<<(ostream & out, const Persoon & p){
    out<<p.naam <<" "<<p.voornaam<<" ("<<p.leeftijd<<")";
    return out;
}


class Groep : public vector<Persoon>{
    public:
        // Hier komt de declaratie van de lidfunctie 'geef_extremum',
        // die een Persoon teruggeeft.
        // Deze lidfunctie krijgt als parameter een functie mee die twee 
        // objecten van de klasse Persoon vergelijkt. 
};

int main(){
    Groep gr;
    gr.push_back(Persoon("Ann","Nelissen",12));
    gr.push_back(Persoon("Bert","Mertens",22));
    gr.push_back(Persoon("Celle","Lauwers",55));
	
    cout<<"Eerste qua naam:     " << gr.geef_extremum(.......);	
    cout<<"Eerste qua voornaam: " << gr.geef_extremum(.......);
    cout<<"Jongste:             " << gr.geef_extremum(.......);
    cout<<"Oudste:              " << gr.geef_extremum(.......);
    return 0;	
}
\end{verbatim}
\end{footnotesize}
\endoef
