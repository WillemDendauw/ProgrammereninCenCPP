\beginoef
Ga verder met de vorige oefening (122).  Vervang het hoofdprogramma door deze code:
\begin{footnotesize}
\begin{verbatim}
int main () {   

   Rechthoek r2(4,6); 
   GekleurdeRechthoek gr1; 
   GekleurdeRechthoek gr3(6,9,"rood"); 
   Vierkant v2(10);

   vector<Rechthoek> v; 
   v.push_back(r2); 
   v.push_back(gr1); 
   v.push_back(gr3); 
   v.push_back(v2);

   for(int i=0 ; i<v.size() ; i++) { 
       cout << v[i]; 
       cout << " oppervlakte: " << v[i].oppervlakte() << endl 
            << " omtrek: " << v[i].omtrek() << endl; 
   }
     
   return 0;
}          
\end{verbatim} 
\end{footnotesize}
De output van dit programma is niet helemaal wat we willen: er is enkel sprake van rechthoeken; de kleuren worden niet uitgeschreven en het vierkant wordt ook niet alsdusdanig herkend.
Pas de code aan, zodat dit wel gebeurt.


\endoef
