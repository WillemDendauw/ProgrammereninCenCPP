\beginoef

Schrijf een hoofdprogramma waarin je alvast volgende code schrijft:
\begin{verbatim}
	double getallen[5] = {5.5,7.7,2.2,9.9,9.8};
	string woorden[3] = {"geloof","hoop","de liefde"};		
	cout << grootste(getallen,5) << endl;
	cout << "De grootste van de drie is " << grootste(woorden,3) << "." << endl;
\end{verbatim}

\begin{enumerate}
\item Schrijf een functie \verb}grootste(array,lengte)} die uit een gegeven array van gegeven lengte het grootste element teruggeeft.
Het type van de elementen die in de array bewaard worden, is niet gekend. 
\\{\bf Tip:} Voeg een functie \verb}grootte(elt)} toe. Deze functie bepaalt de grootte van een element, en moet ge\"\i mplementeerd worden voor elk type waarvoor je de functie \verb}grootste} oproept. De grootte van een woord wordt gedefinieerd als de lengte van het woord.

{\bf Opmerking}: 
\\De functie \verb}grootte(...)} wordt {\bf niet meegeven} aan de functie \verb}grootste(...,...)}: we vragen hier dus geen functiepointers. 

\item Maak in het hoofdprogramma een array van drie personen aan. Elke persoon is van type Persoon (definieer zelf de struct), 
en onthoudt zowel zijn naam (een \verb{string{) als zijn leeftijd (een \verb{int{) en lengte (in meter, dus een \verb{double{).

\item Schrijf een procedure \verb}initialiseer(persoon,naam,leeftijd,lengte)} die de dataleden van een gegeven persoon initialiseert.
Gebruik deze procedure om de drie personen in de array te initialiseren ( vb. Samuel is 12 jaar, lengte is 1m52 / Jente is 22 jaar, lengte is 1m81 / Idris is 42 jaar, lengte 1m73).
\item Schrijf een procedure \verb}print(persoon)} die de gegevens van een persoon uitschrijft op het scherm. (Test uit door een van de personen uit de array uit te schrijven.)
\item Schrijf tenslotte in het hoofdprogramma, de grootste persoon uit, als de grootte van een persoon bepaald wordt door zijn leeftijd.
(Nadien pas je de code aan zodat de grootte van een persoon bepaald wordt door zijn/haar lengte respectievelijk de lengte van zijn/haar naam. Krijg je het verwachte resultaat?)
\end{enumerate}
\endoef
